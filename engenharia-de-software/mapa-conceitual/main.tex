\documentclass[12pt]{article}
\usepackage[brazil]{babel}
\usepackage[utf8]{inputenc}
\usepackage{enumitem}
\usepackage{tikz}
\usetikzlibrary{mindmap, shadows}
\usepackage[a4paper, margin=2cm]{geometry}

\begin{document}

\section*{Perguntas e Respostas}

\subsection*{Aula 10/10}

\begin{enumerate}[label=\textbf{\arabic*.}]
  \item \textbf{Por que Qualidade é um conceito relativo?} \\
   Porque depende da percepção e das necessidades de cada usuário, o que é bom para um pode não ser para outro.

  \item \textbf{Cite e exemplifique 3 dimensões de qualidade.} \\
    \begin{itemize}
      \item \textbf{Funcionalidade:} o software faz o que promete (ex: um app de alarme que dispara na hora certa). 
      \item \textbf{Confiabilidade:} o software mantém funcionamento estável sob condições previstas (ex: não falha durante picos de uso). 
      \item \textbf{Usabilidade:} o software é fácil de usar, aprender e operar (ex: interface intuitiva para novos usuários).
    \end{itemize}

  \item \textbf{Por que é útil ter uma forma de mensurar qualidade?} \\
    Para avaliar, comparar e melhorar o software de forma objetiva e contínua.

  \item \textbf{Por que é necessária uma certificação de qualidade?} \\
    Para garantir que o produto segue padrões reconhecidos e confiáveis, aumentando a credibilidade.

  \item \textbf{Diferencie Produto de Processo.} \\
  Produto é o resultado final (software pronto); processo é o conjunto de atividades que levam à sua criação.

  \item \textbf{Como a \textsc{ISO/IEC} 9126 se subdivide?} \\
    Ela se divide em quatro partes:
    \begin{itemize}
      \item \textbf{Modelo de qualidade:} Composto por seis características principais (funcionalidade, confiabilidade, usabilidade, eficiência, manutenibilidade e portabilidade). 
      \item \textbf{Métricas externas:} Avaliam o comportamento do sistema em operação.
      \item \textbf{Métricas internas:} Avaliam o software antes da entrega, como código-fonte e arquitetura.
      \item \textbf{Métricas de qualidade em uso:} Medem a satisfação e eficiência do usuário final.
    \end{itemize}
\end{enumerate}

\subsection*{Aula 17/10}

\begin{enumerate}[label=\textbf{\arabic*.}]
  \item \textbf{Considerando a \textsc{ISO/IEC} 9126, diferencie qualidade interna e qualidade externa.} \\
    Qualidade interna é medida dentro do código e estrutura do software, enquanto qualidade externa é observada durante o uso real pelo usuário.

  \item \textbf{Diferencie Qualidade de Software de Qualidade de processo de software.} \\
    Qualidade de software avalia o produto final; qualidade de processo foca em como o software é desenvolvido.

  \item \textbf{Cite e explique sucintamente dois “mandamentos do software imaturo” que você já utilizou sem saber durante o curso ou no seu trabalho.} \\
    \begin{itemize}
      \item “Funciona, então está pronto” – usar um programa sem testar bem.
      \item “Corrigir direto no código” – resolver rápido sem analisar o impacto.
    \end{itemize}

  \item \textbf{Considerando os níveis de capacitação \textsc{SPICE}, diferencie “incompleto” de “executado”.} \\
    Incompleto: o processo não é implementado ou é parcialmente feito. \\
    Executado: o processo é realizado e atinge seu propósito básico.

  \item \textbf{O que é o PDCA?} \\
    É um ciclo de melhoria contínua: Planejar (Plan), Executar (Do), Checar (Check) e Agir (Act).
\end{enumerate}

\end{document}
