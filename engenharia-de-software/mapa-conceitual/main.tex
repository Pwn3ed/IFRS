
\documentclass[12pt]{article}
\documentclass[12pt]{article}
\usepackage[brazil]{babel}
\usepackage[utf8]{inputenc}
\usepackage{enumitem}

\begin{document}

\section*{Perguntas e Respostas}

\begin{enumerate}[label=\textbf{\arabic*.}]

  \item \textbf{Por que Qualidade é um conceito relativo?} \\
    Porque o que constitui “bom” depende de quem avalia (usuário, cliente, desenvolvedor), do contexto de uso, dos requisitos implicados e das expectativas, logo o valor de qualidade varia segundo perspectiva e finalidade.

  \item \textbf{Cite e exemplifique 3 dimensões de qualidade.} \\
    \begin{itemize}
      \item \textbf{Funcionalidade:} o software faz o que se espera (ex: um sistema bancário permite transferir valores). \\
      \item \textbf{Confiabilidade:} o software mantém funcionamento estável sob condições previstas (ex: não falha durante picos de uso). \\
      \item \textbf{Usabilidade:} o software é fácil de usar, aprender e operar (ex: interface intuitiva para novos usuários).
    \end{itemize}

  \item \textbf{Por que é útil ter uma forma de mensurar qualidade?} \\
    Porque sem métricas não se sabe se os objetivos de qualidade foram atingidos, não se identifica onde melhorar, não se compara opções ou produtos, e não se toma decisões informadas para gerenciar riscos.

  \item \textbf{Por que é necessária uma certificação de qualidade?} \\
    Porque atesta que processos ou produtos seguem padrões reconhecidos, passa confiança a clientes interessados, garante consistência, transparência e redução de erros ou falhas, além de favorecer concorrência e conformidade.

  \item \textbf{Diferencie Produto de Processo.} \\
    \begin{itemize}
      \item \textbf{Produto:} o resultado final ou item entregue (ex: software, módulo funcional). \\
      \item \textbf{Processo:} o conjunto de atividades, métodos e recursos usados para produzir o produto (ex: ciclo de desenvolvimento, revisão de código, testes).
    \end{itemize}

  \item \textbf{Como a \textsc{ISO/IEC} 9126 se subdivide?} \\
    Ela se divide em quatro partes: \\
    \begin{itemize}
      \item Parte\,1: Modelo de qualidade (características e subcaracterísticas) \\
      \item Parte\,2: Métricas externas \\
      \item Parte\,3: Métricas internas \\
      \item Parte\,4: Métricas de qualidade em uso (“quality in use”)
    \end{itemize}

  \item \textbf{Considerando a \textsc{ISO/IEC} 9126, diferencie qualidade interna e qualidade externa.} \\
    \begin{itemize}
      \item \textbf{Qualidade interna:} refere-se a atributos do software em estado não executável ou sob análise estática (ex: estrutura do código-fonte). \\
      \item \textbf{Qualidade externa:} refere-se ao comportamento do software quando em execução, em seu ambiente de sistema (ex: resposta, falhas, interoperabilidade).
    \end{itemize}

  \item \textbf{Diferencie Qualidade de Software de Qualidade de processo de software.} \\
    \begin{itemize}
      \item \textbf{Qualidade de software:} refere-se ao produto de software — suas funcionalidades, desempenho, confiabilidade, manutenibilidade etc. \\
      \item \textbf{Qualidade de processo de software:} refere-se à forma como o software foi desenvolvido — organização, métodos, controle, repetibilidade — e como isso influencia no produto final.
    \end{itemize}

  \item \textbf{Cite e explique sucintamente dois “mandamentos do software imaturo” que você já utilizou sem saber durante o curso ou no seu trabalho.} \\
    Dois exemplos: \\
    \begin{itemize}
      \item “Não utilizar uma equipe de teste independente”: assume-se que os programadores testam seu próprio código, o que reduz esforço de QA mas pode comprometer imparcialidade e cobertura. \\
      \item “Afastar o cliente do processo de desenvolvimento”: o cliente não participa ativamente após os requisitos iniciais, o que pode levar a resultados desalinhados com suas reais necessidades.
    \end{itemize}

  \item \textbf{Considerando os níveis de capacitação \textsc{SPICE}, diferencie “incompleto” de “executado”.} \\
    No modelo \textsc{ISO/IEC} 15504 (ou SPICE) temos: \\
    \begin{itemize}
      \item Nível 0 (“Incompleto”): o processo não é implementado ou falha em atingir seu propósito. \\
      \item Nível 1 (“Executado” ou “Performed”): o processo é implementado e atinge seu propósito, mas de forma não sistemática ou padronizada.
    \end{itemize}

  \item \textbf{O que é o PDCA?} \\
    O PDCA (Plan-Do-Check-Act) é um ciclo de melhoria contínua: \\
    \begin{itemize}
      \item \textbf{Plan:} planejar. \\
      \item \textbf{Do:} executar. \\
      \item \textbf{Check:} verificar/checar os resultados. \\
      \item \textbf{Act:} agir/corrigir para promover melhorias no processo ou produto.
    \end{itemize}

\end{enumerate}

\end{document}
\usepackage[brazil]{babel}
\usepackage[utf8]{inputenc}
\usepackage{enumitem}

\begin{document}

\section*{Perguntas e Respostas}

\begin{enumerate}[label=\textbf{\arabic*.}]

  \item \textbf{Por que Qualidade é um conceito relativo?} \\
    Porque o que constitui “bom” depende de quem avalia (usuário, cliente, desenvolvedor), do contexto de uso, dos requisitos implicados e das expectativas, logo o valor de qualidade varia segundo perspectiva e finalidade.

  \item \textbf{Cite e exemplifique 3 dimensões de qualidade.} \\
    \begin{itemize}
      \item \textbf{Funcionalidade:} o software faz o que se espera (ex: um sistema bancário permite transferir valores). \\
      \item \textbf{Confiabilidade:} o software mantém funcionamento estável sob condições previstas (ex: não falha durante picos de uso). \\
      \item \textbf{Usabilidade:} o software é fácil de usar, aprender e operar (ex: interface intuitiva para novos usuários).
    \end{itemize}

  \item \textbf{Por que é útil ter uma forma de mensurar qualidade?} \\
    Porque sem métricas não se sabe se os objetivos de qualidade foram atingidos, não se identifica onde melhorar, não se compara opções ou produtos, e não se toma decisões informadas para gerenciar riscos.

  \item \textbf{Por que é necessária uma certificação de qualidade?} \\
    Porque atesta que processos ou produtos seguem padrões reconhecidos, dá confiança a clientes/parte interessadas, assegura consistência, transparência e redução de erros ou falhas, além de favorecer concorrência e conformidade.

  \item \textbf{Diferencie Produto de Processo.} \\
    \begin{itemize}
      \item \textbf{Produto:} o resultado final ou item entregue (ex: software, módulo funcional). \\
      \item \textbf{Processo:} o conjunto de atividades, métodos e recursos usados para produzir o produto (ex: ciclo de desenvolvimento, revisão de código, testes).
    \end{itemize}

  \item \textbf{Como a \textsc{ISO/IEC} 9126 se subdivide?} \\
    Ela se divide em quatro partes: \\
    \begin{itemize}
      \item Parte\,1: Modelo de qualidade (características e subcaracterísticas) \\
      \item Parte\,2: Métricas externas \\
      \item Parte\,3: Métricas internas \\
      \item Parte\,4: Métricas de qualidade em uso (“quality in use”)
    \end{itemize}

  \item \textbf{Considerando a \textsc{ISO/IEC} 9126, diferencie qualidade interna e qualidade externa.} \\
    \begin{itemize}
      \item \textbf{Qualidade interna:} refere-se a atributos do software em estado não executável ou sob análise estática (ex: estrutura do código-fonte). \\
      \item \textbf{Qualidade externa:} refere-se ao comportamento do software quando em execução, em seu ambiente de sistema (ex: resposta, falhas, interoperabilidade).
    \end{itemize}

  \item \textbf{Diferencie Qualidade de Software de Qualidade de processo de software.} \\
    \begin{itemize}
      \item \textbf{Qualidade de software:} refere-se ao produto de software — suas funcionalidades, desempenho, confiabilidade, manutenibilidade etc. \\
      \item \textbf{Qualidade de processo de software:} refere-se à forma como o software foi desenvolvido — organização, métodos, controle, repetibilidade — e como isso influencia no produto final.
    \end{itemize}

  \item \textbf{Cite e explique sucintamente dois “mandamentos do software imaturo” que você já utilizou sem saber durante o curso ou no seu trabalho.} \\
    Dois exemplos: \\
    \begin{itemize}
      \item “Não utilizar uma equipe de teste independente”: assume-se que os programadores testam seu próprio código, o que reduz esforço de QA mas pode comprometer imparcialidade e cobertura. \\
      \item “Afastar o cliente do processo de desenvolvimento”: o cliente não participa ativamente após os requisitos iniciais, o que pode levar a resultados desalinhados com suas reais necessidades.
    \end{itemize}

  \item \textbf{Considerando os níveis de capacitação \textsc{SPICE}, diferencie “incompleto” de “executado”.} \\
    No modelo \textsc{ISO/IEC} 15504 (ou SPICE) temos: \\
    \begin{itemize}
      \item Nível 0 (“Incompleto”): o processo não é implementado ou falha em atingir seu propósito. \\
      \item Nível 1 (“Executado” ou “Performed”): o processo é implementado e atinge seu propósito, mas de forma não sistemática ou padronizada.
    \end{itemize}

  \item \textbf{O que é o PDCA?} \\
    O PDCA (Plan-Do-Check-Act) é um ciclo de melhoria contínua: \\
    \begin{itemize}
      \item \textbf{Plan:} planejar. \\
      \item \textbf{Do:} executar. \\
      \item \textbf{Check:} verificar/checar os resultados. \\
      \item \textbf{Act:} agir/corrigir para promover melhorias no processo ou produto.
    \end{itemize}

\end{enumerate}

\end{document}
