\documentclass{article}
\usepackage[utf8]{inputenc}
\usepackage[brazil]{babel}
\usepackage{multirow}
\usepackage{array}
\usepackage[table]{xcolor}
\usepackage{geometry}
\usepackage{float}
\usepackage{xparse}

\geometry{paperwidth=700pt,paperheight=1400pt, margin=1cm}

\definecolor{lightgray}{gray}{0.9}
\definecolor{blueval}{RGB}{0,0,180}
\definecolor{greenval}{RGB}{0,150,0}

\newcommand{\hoje}{08/10/2025}

\newcommand{\canvasTitle}{}
\newcommand{\canvasDescription}{}
\newcommand{\canvasCaption}{}
\newcommand{\canvasPrimeiroBloco}{}
\newcommand{\canvasSegundoBloco}{}
\newcommand{\canvasTerceiroBloco}{}
\newcommand{\canvasQuartoBloco}{}

\newcommand{\setcanvas}[5]{%
  \def\canvasCaption{#1}%
  \def\canvasPrimeiroBloco{#2}%
  \def\canvasSegundoBloco{#3}%
  \def\canvasTerceiroBloco{#4}%
  \def\canvasQuartoBloco{#5}%
}

\newcommand{\appendto}[2]{%
  \expandafter\gdef\expandafter#1\expandafter{#1 #2}%
}

\newcommand{\canvas}[3]{%
  \renewcommand{\arraystretch}{1.8}
  \section*{#1}
  #2
  \vspace{0.5cm}

  \begin{table}[H]
  \centering
  \caption{\canvasCaption}
  \setlength{\tabcolsep}{4pt}
  \begin{tabular}{|p{6cm}|p{6cm}|p{6cm}|}
  \hline
  \rowcolor{lightgray}
  \textbf{Parceiros Chave} & \centering\textbf{PROPOSTA DE VALOR} & \textbf{Relacionamento com Clientes} \\
  \cline{1-3}
  \canvasPrimeiroBloco
  \hline
  \rowcolor{lightgray}
  \textbf{Atividades Chave} & & \textbf{Canais} \\
  \cline{1-3}
  \canvasSegundoBloco
  \hline
  \rowcolor{lightgray}
  \textbf{Recursos Chave} & & \textbf{Segmentos de Clientes} \\
  \cline{1-3}
  \canvasTerceiroBloco
  \hline
  \rowcolor{lightgray}
  \textbf{Estrutura de Custos} & & \textbf{Fluxo de Receitas} \\
  \hline
  \canvasQuartoBloco
  \hline
  \end{tabular}

  \begin{center}
  \ifx#3\empty\else
  \textbf{Legenda:} \\
  #3
  \fi
  \end{center}
  \end{table}
}

\newcommand{\canvasHeader}[3]{%
\begin{table}[H]
\centering
\renewcommand{\arraystretch}{1.3}
\setlength{\tabcolsep}{10pt}
\rowcolors{1}{lightgray}{white}
  \begin{tabular}{|p{4cm}|p{7cm}|p{3cm}|p{2cm}|}
    \hline
    \textbf{Projeto} & \textbf{Criado Por} & \textbf{Data} & \textbf{Versão} \\
    \hline
    #1 & #2 & \hoje & #3 \\
    \hline
  \end{tabular}
\end{table}
}


\begin{document}

\canvasHeader{ColabHUB}{Daniel, Diego, Gustavo Martins, Gustavo Policarpo e Leonardo}{1.0}

\setcanvas
{Modelo de negócio de uma plataforma para colaboração em projetos para desenvolvedores e entusiastas}
{
• Empresas de Tecnologia   & • Networking   & • Website \\
• Provedores de Hospedagem & • Conveniência & • Redes Sociais \\
                           & • Colaboração  & • Aplicativo Móvel \\
                           & • Estudo       & \\
}
{
• Desenvolvimento da Plataforma     & & • Google Adwords \\
• Promoção da Plataforma nos Canais & & • Facebook Ads \\
• Busca por Parceiros               & & • Email Marketing \\
                                    & & • Afiliados (Influenciadores) \\
                                    & & • Social Orgânico \\
                                    & & • Blog \\
}
{
• Aquisição de Equipamentos & & • Desenvolvedores \\
• Equipe de desenvolvimento & & • UI/UX Designers \\
• Contratação de Servidor   & & • Empreendedores \\
• Compra de Domínio         & & • Startups \\
• Verba para Marketing      & & \\
}
{
• Hospedagem                                                         & & • Planos de Assinatura \\
• Domínio                                                            & & • Publicidade de Parceiros \\
• Marketing                                                          & & • Patrocinadores \\
• Manutenção                                                         & & \\
• Serviços Externos (API de pagamentos, integração com repositórios) & & \\
}

\canvas{Canvas Inicial}
{O projeto \textbf{ColabHUB} tem como proposta conectar desenvolvedores, empreendedores e empresas em um ambiente colaborativo, onde ideias podem se transformar em projetos reais. Esta primeira versão do Canvas apresenta a visão inicial do grupo, ainda sem validações externas.}
{}

\newpage

\appendto{\canvasPrimeiroBloco}{
\textcolor{red}{• Instituições de ensino.} &
\textcolor{red}{• Simular o mercado de trabalho real.} &
\textcolor{red}{• Fórum interno para se comunicar dentro do projeto.} \\
& \textcolor{red}{• Criação de Portfólio} & \\
}

\appendto{\canvasSegundoBloco}{
& & \textcolor{red}{• Divulgar em lugares que desenvolvedores realmente usam como Linkedin e Discord.} \\
}

\appendto{\canvasTerceiroBloco}{
\textcolor{red}{• Coordenador para orientar e direcionar o marketing} &
& 
\textcolor{red}{• Behance, plataforma para encontrar portfólios} \\
\textcolor{red}{• Iniciar utilizando serviços gratuitos ou baratos, como Vercel} &
& 
\textcolor{red}{• Começar focando em desenvolvedores por já haver um conexões com esta área.} \\
}

\appendto{\canvasQuartoBloco}{
\textcolor{red}{• Monday, plataforma para gerenciar custos e organizar o fluxo de trabalho.} &
&
\textcolor{red}{• Consultoria de especialistas da área para auxiliar no projeto.} \\
}

\canvas{Primeira Validação}
{
Após uma conversa com um especialista técnico, foram identificados pontos de melhoria voltados à parte operacional e estrutural do projeto. Essa validação ajudou a ajustar os recursos e atividades principais para garantir a viabilidade do ColabHUB.
}
{
\textcolor{red}{\textbf{Vermelho}} = Primeira validação
}

\newpage

\appendto{\canvasPrimeiroBloco}{
\textcolor{blueval}{• Plataforma de contratação.} &
\textcolor{blueval}{• Fomentar o trabalho em equipe.} &
\textcolor{blueval}{• Criar formas de manter contato frequente e útil com usuários.} \\
\textcolor{blueval}{• Priorizar parcerias com empresas de tecnologia da região} & & \\
}

\appendto{\canvasSegundoBloco}{
\textcolor{blueval}{• Chat para comunicação interna.} &
& 
\textcolor{blueval}{• Comunidade do Whatsapp} \\
}

\appendto{\canvasTerceiroBloco}{
\textcolor{blueval}{• Contratação de gerente para orientar o crescimento da empresa.} &
& 
\textcolor{blueval}{• Estudantes} \\
}

\appendto{\canvasQuartoBloco}{
\textcolor{blueval}{• Métricas e acompanhamento.} &
&
\textcolor{blueval}{• Assinatura.} \\
& & \textcolor{blueval}{• Limitar a quantidade de colaboradores em planos gratuitos}  \\
}

\canvas{Segunda Validação}
{
Após a análise de um especialista com uma visão mais comercial, foram feitas melhorias no posicionamento do ColabHUB, focando em estratégias de mercado, atração de público e sustentabilidade financeira.
}
{
\textcolor{red}{\textbf{Vermelho}} = Primeira validação \quad
\textcolor{blueval}{\textbf{Azul}} = Segunda validação
}

\newpage

\appendto{\canvasPrimeiroBloco}{
}

\appendto{\canvasSegundoBloco}{
}

\appendto{\canvasTerceiroBloco}{
}

\appendto{\canvasQuartoBloco}{
}

\canvas{Terceira Validação}
{
Após a aplicação de uma pesquisa com o público-alvo, confirmou-se que o ColabHUB é um projeto relevante. Esta etapa consolidou as ideias anteriores e direcionou os próximos passos de acordo com o que o público mais valorizou.
}
{
\textcolor{red}{\textbf{Vermelho}} = Primeira validação \quad
\textcolor{blueval}{\textbf{Azul}} = Segunda validação \quad
\textcolor{greenval}{\textbf{Verde}} = Terceira validação
}

\end{document}
