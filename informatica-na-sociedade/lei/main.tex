\documentclass[12pt, a4paper]{article}
\usepackage[brazil]{babel}
\usepackage[utf8]{inputenc}
\usepackage[T1]{fontenc}
\usepackage[margin=3cm]{geometry}
\usepackage{indentfirst}
\usepackage{setspace}
\usepackage{enumitem}
\usepackage{titlesec}

% Configurações ABNT
\onehalfspacing
\setlength{\parindent}{1.25cm}
\geometry{a4paper, left=3cm, right=2cm, top=3cm, bottom=2cm}

% Formatação de seções
\titleformat{\section}{\bfseries\large}{\thesection}{1em}{}
\titlespacing{\section}{0pt}{12pt}{6pt}

\begin{document}

\begin{center}
    {\large \textbf{RESUMO DE LEI / MARCO LEGAL}}
\end{center}

\section{Título da Lei / Marco Legal:}
Lei de Propriedade Industrial (LPI) - Principal marco legal da propriedade industrial brasileira

\section{Número e Ano:}
Lei nº 9.279, de 14 de maio de 1996

\section{Objetivo Principal:}
Regular direitos e obrigações relativos à propriedade industrial, protegendo as criações de aplicação comercial para incentivar a inovação tecnológica e o desenvolvimento econômico do país, assegurando aos criadores o direito de propriedade sobre suas invenções.

\section{Principais Pontos / Artigos Relevantes:}
\begin{enumerate}[label=\textbf{Ponto \arabic*:}, leftmargin=*, align=left]
    \item \textbf{Proteção por Patentes} - Concede patentes para invenções (20 anos) e modelos de utilidade (15 anos) que atendam aos requisitos de novidade, atividade inventiva e aplicação industrial (Art. 8º ao 44)
    
    \item \textbf{Proteção de Marcas} - Garante o registro e uso exclusivo de sinais distintivos para produtos e serviços em todo território nacional (Art. 122 ao 207)
    
    \item \textbf{Proteção de Desenho Industrial} - Assegura o direito sobre o aspecto ornamental de objetos, protegendo seu design estético (Art. 95 ao 121)
    
    \item \textbf{Repressão a Práticas Ilícitas} - Combate falsas indicações geográficas, concorrência desleal e prevê crimes contra a propriedade industrial com penas de reclusão e multa (Art. 207 ao 227)
\end{enumerate}

\section{Aplicações Práticas / Exemplos:}
\begin{itemize}
    \item \textbf{Setor de Tecnologia}: Registro de marcas de aplicativos e softwares para proteger a identidade do negócio
    \item \textbf{Indústria Farmacêutica}: Proteção de novos medicamentos através de patentes, impedindo cópias não autorizadas
    \item \textbf{Comércio}: Registro de marcas de estabelecimentos comerciais para diferenciação no mercado
\end{itemize}

\section{Impacto ou Importância da Lei:}
\begin{itemize}
    \item \textbf{Econômico}: Setores intensivos em DPI geraram 50,2\% do valor adicionado total na economia brasileira (2020-2022)
    \item \textbf{Inovação Tecnológica}: Estimula investimentos em pesquisa e desenvolvimento ao garantir retorno sobre o investimento em inovação
    \item \textbf{Desenvolvimento Nacional}: Fortalece a competitividade das empresas brasileiras no mercado internacional
    \item \textbf{Segurança Jurídica}: Oferece ambiente previsível para investimentos em criação e desenvolvimento de novos produtos e tecnologias
    \item \textbf{Combate à Pirataria}: Instrumento legal para combater falsificações e concorrência desleal
\end{itemize}

\section{Referências / Fontes Consultadas:}

BRASIL. Lei nº 9.279, de 14 de maio de 1996. Regula direitos e obrigações relativos à propriedade industrial. Diário Oficial da União, Brasília, DF, 14 maio 1996.

BRASIL. Lei nº 9.610, de 19 de fevereiro de 1998. Altera, atualiza e consolida a legislação sobre direitos autorais. Diário Oficial da União, Brasília, DF, 19 fev. 1998.

MINISTÉRIO DO DESENVOLVIMENTO, INDÚSTRIA, COMÉRCIO E SERVIÇOS. Estudo sobre o impacto econômico dos direitos de propriedade intelectual no Brasil. Brasília: MDIC, 2023.

TRIBUNAL DE JUSTIÇA DO DISTRITO FEDERAL. Propriedade Intelectual: conceitos e aplicações. Brasília: TJDFT, 2022.

\section{Estudante responsável:}

Diego Michel Prestes

6º semestre - Analise e Desenvolvimento de Sistemas

\end{document}
